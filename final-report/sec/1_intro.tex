\section{Introduction}

In this project, we aim to address the challenge of classifying paintings according to their artistic styles. This task lies at the intersection of computer vision, machine learning, and art history, making it a multifaceted problem with significant implications for both technological and cultural domains\cite{CETINIC2018107}. In recent years, the automated classification of art painting styles using deep convolutional neural networks (CNNs) has become essential for analyzing and categorizing vast digitized art collections. These models can learn hierarchical visual features directly from raw image data, providing a powerful tool for art analysis\cite{li2025enhanced}\cite{imran2023artistic}. However, accurately recognizing and distinguishing stylistic characteristics across diverse art movements remains a challenge due to high intra-class variability and inter-class similarity\cite{alkofer2021using}. Developing robust classification methods is critical for scalable digital archiving, enhancing curatorial workflows, and enabling large-scale quantitative analysis of artistic trends to support art historical research.

Our project builds upon the approach proposed in\cite{imran2023artistic}, which involves a two-stage architecture combining a deep neural network (DNN) and a shallow neural network (SNN) adapter. The DNN acts as a feature extractor, while the SNN adapter is responsible for decision-making. Each input image is divided into five distinct patches: top-left, top-right, bottom-left, bottom-right, and center. The DNN independently classifies each of these five patches, and the SNN adapter aggregates the results to produce a final classification. This architecture allows for a more detailed examination of different regions within an artwork, capturing fine-grained information and preserving important artistic details\cite{imran2023artistic}. Importantly, the SNN operates independently of the DNN, meaning that the introduction of the SNN does not alter the architecture or weights of the DNN. This independence allows the SNN to function as a flexible adapter, enhancing the classification process without imposing any architectural constraints on the DNN. Thus, the integrity and performance of the original DNN are maintained while adding an additional layer of refinement to the classification results.

By comparing the prediction results of the DNN direct output and the SNN adapter output, we observed that while the SNN achieves higher overall accuracy, the DNN direct prediction exhibits superior accuracy in certain specific classes. To integrate the strengths of both networks, we proposed a hierarchical architecture. We trained an SNN adapter using four image patches: left-top, left-bottom, right-top, and right-bottom. The final prediction is then calculated as a weighted sum of the DNN's direct prediction and the SNN adapter's output. This combined approach not only yields higher total accuracy than the original SNN adapter architecture but also results in more stable accuracy across different classes.

Furthermore, we also explored how the selection of patches affects the performance of the model. [To be continued.]

Building on the above methodology, our project makes the following key contributions: 
\begin{itemize}
    \item We implement DNN + SNN sdapter architecture proposed in \cite{imran2023artistic}, proof its effectiveness on \text{DenseNet-121, VGG-19, ResNet-50} DNN architectures by ablation study.
    \item We propose a hierarchical architecture that combines the strengths of both the DNN and SNN adapter, resulting in improved accuracy and stability across different classes.
    \item We conduct an extensive analysis of the impact of different patch selections on the model's performance, providing insights into the optimal configuration for art style classification.
\end{itemize}
