\section{Problem Statement} \label{sec:statement}
In this project, we address the problem of classifying paintings according to
their artistic styles. Given the vast number of possible artistic styles, we
focus on five major modern styles: \textbf{Cubism}, \textbf{Impressionism},
\textbf{Expressionism}, \textbf{Realism}, and \textbf{Abstract}.
\paragraph{Dataset.}
We utilize the Painter by Numbers dataset \cite{painter_by_numbers} as the
primary source of paintings for our study. There are 103250 unique paintings in
total, most of which are from WikiArt.org. Together with the images, the
dataset provides a .csv file containing all the informations about each
painting, including artist, date, content genre, size, source, title,
\textbf{style}, \textbf{group}(for training or for testing), and its
\textbf{filename} in the dataset. Only the last three terms are needed for our
project.

The paintings in the dataset encompass a wide variety of styles with highly
granular categorizations. For clarity and consistency with the baseline model,
we consolidated these into five principal categories listed above. The specific
style correspondences are detailed in Table 1. A small number of artworks not
labeled under any of these designated categories were excluded during data
processing.
\begin{table}[ht]
    \centering
    \small
    % Requires \usepackage{booktabs}
    % Table width = 0.8 of half page = 0.4\textwidth
    \begin{tabular}{@{}p{0.16\textwidth} p{0.24\textwidth}@{}}
        \toprule
        \textbf{Style Category} & \textbf{Specific Styles}         \\
        \midrule
        Cubism                  & Cubism, Tubism,                  \\
                                & Cubo-Expressionism,              \\
                                & Mechanistic Cubism,              \\
                                & Analytical Cubism,               \\
                                & Cubo-Futurism,                   \\
                                & Synthetic Cubism                 \\
        \midrule
        Impressionism           & Impressionism,                   \\
                                & Post-Impressionism,              \\
                                & Synthetism, Divisionism,         \\
                                & Cloisonnism                      \\
        \midrule
        Expressionism           & Expressionism,                   \\
                                & Neo-Expressionism,               \\
                                & Figurative Expressionism,        \\
                                & Fauvism                          \\
        \midrule
        Realism                 & Realism, Hyper-Realism,          \\
                                & Photorealism, Naturalism,        \\
                                & Analytical Realism               \\
        \midrule
        Abstract                & Abstract Art, New Casualism,     \\
                                & Post-Minimalism, Orphism,        \\
                                & Constructivism, Lettrism,        \\
                                & Neo-Concretism, Suprematism,     \\
                                & Spatialism, Conceptual Art,      \\
                                & Tachisme, Neoplasticism,         \\
                                & Post-Painterly Abstraction,      \\
                                & Precisionism, Hard Edge Painting \\
        \bottomrule
    \end{tabular}
    \caption{Mapping of Art Style Categories}
    \label{tab:style_mapping_halfwidth_80}
\end{table}

% (TODO: Describe the preprocessing
% steps, including how multiple style labels are consolidated into the five
% selected categories; explain the preparation of input data for the shallow
% neural network.)
\paragraph{Evaluation.}
Classification accuracy is adopted as the primary evaluation metric to assess
model performance.
\paragraph{Baseline.}
In this study, we adopt the ArtNet model \cite{artnet} as our baseline for
comparative analysis. ArtNet is specifically designed to classify paintings
into five distinct modern art styles: Cubism, Impressionism, Expressionism,
Realism, and Abstract. A key feature of ArtNet is its preprocessing strategy,
where each input image is first padded to a uniform size and then divided into
five patches. Both the full image and its patches are included in the training
set, enabling the model to learn from global composition as well as local
texture details such as brush strokes and color tones. \\ This dual-level
learning approach aligns well with our patch-based decision-making framework,
making ArtNet an appropriate choice for our baseline model.
\paragraph{Expected Results.}
We expect the following improvements at different parts of the project: \begin{itemize} \item \textbf{Implementation part:} By introducing a shallow neural network as a decision-making module that aggregates the predictions from individual patches, we expect the overall classification accuracy to surpass that of the baseline model. \item \textbf{Optimization part:} \begin{itemize} \item By adopting a more strategic or adaptive approach for selecting the fifth
                    patch—such as using a downsampled version of the entire image or extracting a
                    patch containing the main semantic content—we anticipate achieving higher
                    classification accuracy compared to simply selecting the center patch. \item By employing a more sophisticated hierarchical structure for patch extraction,
                    rather than independently classifying five uniformly sized patches, we aim to
                    further improve accuracy and provide insights into the spatial importance of
                    different regions within a painting. \end{itemize} \end{itemize}